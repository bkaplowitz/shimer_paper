\documentclass{article}
\usepackage{booktabs}
\usepackage{dcolumn}
\usepackage[utf8]{inputenc}
\usepackage{subcaption}
\usepackage{amsmath}
\usepackage{amsfonts}
\usepackage{amssymb}
\usepackage{amsbsy}
\begin{document}
\title{Increasing Returns and Long-Run Growth}

\author[1]{Brandon Kaplowitz, Eric Li}
\maketitle

\begin{abstract}
This paper extends the Long-run growth model with increasing returns on knowledge first presented by Romer (1986). We introduce an additional factor of production, human capital, which exhibits decreasing returns. This additional factor enables us to endogenize the limit on the rate of marginal return of the research technology, which is bounded exogenously by Romer. Moreover, as human capital may be measured empirically by proxy variables such as education, we are able to test theory with data. Our extension does not affect the increasing long-run growth result derived by Romer.  

\end{abstract}

\section{Introduction}
A basic assumption of the neoclassical growth model is diminishing marginal return on capital, which produces constant growth rates at steady state. Romer (1986) relaxes this assumption and presents a model of long-run growth in which knowledge is assumed to be a factor of production that exhibits increasing marginal productivity. To guarantee the existence of a competitive equilibrium, Romer endogenizes the growth of knowledge by introducing a research technology, an investment function which produces knowledge. By imposing an exogenous limit on this research function, Romer derives a competitive equilibrium with increasing long-run growth rates. 

Romer's model is supported, at least qualitatively, by many empirical evidence. As Romer points out, long-run growth rates of frontier nations has increased from around 0\% to today's 2\%, suggesting increasing growth. Marginal knowledge also seems to exhibit increasing productivity, as evidenced by Moore's Law. It is, nevertheless, a bare-bone model with too-powerful assumptions. For instance, all factors of production other than knowledge is assumed to be in fixed supply. Moreover, Romer imposes an exogenous limit on the research technology, an assumption that may be endogenized. This model also does not lend itself to direct empirical testing, as knowledge, the only variable input, may be difficult to quantify or measure. 

In this paper, we first present Romer's long-run growth model in section II. While the original model is solved as a continuous-time problem, We choose to discretize the model for ease of empirical testing in latter parts of the paper. In section III, we introduce an additional variable in the production function, namely human capital. Moreover, we endogenize the limit on the research technology using the rate of return on human capital, a decreasing value. In section IV, we introduce ways to measure human capital and knowledge, and presents methodologies to test the extended model. We present our results in section V.
 
\section{Discretized Model}

We first present Romer's original model in a discretized form. We assume an economy characterized by N identical representative consumers and N different firms. We let the number of consumers to be equal to the number of firms for algebraic ease, such as per capita and per firm values are always equal. This same assumption is implicitly made in the neoclassical model, which normalizes the number of consumers and firms to 1. We allow for multiple firms because we wish to introduce a distinction between private knowledge, $k_i$, which firm $i$ augments by investment, and public knowledge $K$, which every firm has access to and takes as given. Public knowledge is defined simply as the sum of all private knowledge, i.e., $K \equiv \sum\limits_{n=1}^{N} k_i$. Implicitly, perfect information is assumed as every firm is able to "see" the aggregate knowledge about production. This is not unrealistic, considering many regulations require firms to disclose relevant information, and that even patented knowledge cannot be kept perfectly secret. The important implication is that knowledge has a positive external effect on production. It is true that dissemination of information is often imperfect in real life, however, it is evident that knowledge is a public good, and this assumption allows us to easily model this externality. Another assumption is that the initial knowledge stock $k_{i,0}$ is constant across all firms. This assumption may initially sound very strong, but its rationale will be discussed a few paragraphs below. 

At each period $t$, each consumer derives utility from consumption. Their utility function satisfy the usual constraints of differentiability and convexity. Each firm seeks to maximize its profit following a production function $F(k_{i,t},K_t,x_{i,t})$, where $x_{i,t}$ represent a vector of inputs of production other than private and aggregate knowledge. Romer assumes that these factors are supplied inelastically at a fixed level, as he is mainly concerned with analyzing increasing returns of knowledge. We will later relax this assumption and include human capital, one of the factor internalized in $x_{i,t}$, in the model. 

Romer makes two key assumptions regarding the production function $F$. First, F exhibit global increasing marginal productivity of knowledge from a social point of view. This is to say, for any fixed $x$, the per firm production function $F(k,K,x) = F(k,Nk,x)$ is convex from the perspective of a social planner who is able to set the level of $k$. Note that aggregate production is likewise convex as all firms face the same production function. This assumption distinguishes the model from that of neoclassical growth. The second key assumption is that for any fixed value of $K$, $F$ is concave as a function of $k$ and $x$. In other words, $F$ exhibits diminishing returns with respect to all factors of production other than $K$. The assumption that certain factors of production faces diminishing returns is the same as that given in the neoclassical model. As Romer proves mathematically, this concavity condition is necessary for the existence of a competitive equilibrium. With the assumption of concavity, Romer then assumes that F is homogeneous of degree one as a function of $k$ and $x$ when $K$ is held constant. No generality is lost according to a proof cited by Romer, first given by Rockafellar (1970). Mathematically, as $F$ is homogeneous of degree 1 in $k$ and $x$ and that $F$ is increasing in $K$, we know that for any $\Psi > 1$,
\[F(\Psi k, \Psi K, \Psi x) > F(\Psi k, K, \Psi x) = F(k,K,x)\]

Each firm also faces an investment function, through which additional knowledge may be produced by forgoing current produced goods. Unlike in the neoclassical model, however, there is no depreciation of private knowledge and most importantly, this trade-off is not assumed to be one-to-one. 
By investing $I$ units of good in research, a firm with a current stock of private knowledge $k$ will reap $G(I,k)$ additional unit of knowledge. The function $G$ is assumed to be concave and homogeneous of degree one. As $G(I,k) = kG(I,1)$, we can define $g(I) = G(I,1)$ and say that the rate of growth of knowledge is $kg(I)$. 

A crucial assumption about $G$ and $g$ is that $g$ is bounded from above by some exogenous constant $\alpha$. This imposes strong diminishing returns in
research. Given the stock of private knowledge, the marginal product
of additional investment in research, $Dg$, falls so rapidly that $g$ is
bounded even if it is increasing. This is to say, research always yields additional knowledge, but this becomes more and more difficult as knowledge begins to accumulate. This is not unrealistic, as more and more resources need to be devoted for the newest technological breakthrough. This assumption is crucial in guaranteeing the existence of a equilibrium, as it counteracts with the increasing returns of knowledge to make sure productivity does not grow without bound. Another inessential but natural assumption is that $g$ is bounded from below by the value $g(0) = 0$. As knowledge does not depreciate, zero research implies zero change in k; moreover, existing knowledge cannot be converted back into consumption goods. As a normalization to fix the unit of knowledge, $Dg(0) = 1$ is assumed. 

Note that all firms face the same production function and investment function and are thus distinguished only by their private knowledge. Because we assumed that they all possess the same initial knowledge stock, it is clear that all firms will always make the same choices have have the same quantities of private knowledge at all times. Note, this does not necessarily mean all firms possess the same knowledge; rather, all firms possess the same amount of "effective-knowledge." Nevertheless, as $k_{i,t}$ is the same for all $i$, we can conveniently drop all subscripts $i$ and treat the firms as mathematically identical. This is the exact assumption made in the neoclassical growth. Recall that we derived the condition of identical firms from the assumption that initial knowledge stock $k_{i,0}$ is the same for all $i$. It is now apparent that this is not an extraordinary assumption as it is analogous to the normalization of identical firms to one representative firm with one initial capital stock $K_0$ in the neoclassical model. 

In his paper, Romer does not state nor solve the competitive equilibrium. Citing a proof from an earlier paper (Romer 1983), Romer showed that the equilibrium allocations are the same as those derived from a social planner's problem. Thus, instead seeking private solution, Romer simply eliminates prices and solves the following familiar problem, presented here in discretized, rather than the original, continuous-time, form: 
\begin{equation*}
\begin{aligned}
& \underset{\{c_t\}}{\text{max}}
& & \sum\limits_{n=0}^{\infty} \beta^{t} U(c_t) \\
& \text{subject to}
& & k_{t+1} = k_t(1+g(F(k_t, K_t, \textbf{x}) - c_t))
\end{aligned}
\end{equation*}
   
Note that as all consumers are identical, the maximization of social welfare is simply the maximization of a representative individual's life time discounted utility. The constraint is simply the law of motion of knowledge: the stock of knowledge in the next period is equal to the current knowledge stock plus new knowledge created by investment in research. As $g$ is an increasing function, we see that forgoing consumption goods leads to higher investments, and greater knowledge in the next period. Note that in order to solve this problem, the economy-wide goods clearing condition must also be met:


To illustrate different behaviors possible with this increasing returns model, Romer employed three examples. He assumes the following functional form for the production function:
\[f(k,K) = k^vK^\gamma = N^\gamma k^{v+\gamma}\]

The non-increasing marginal returns of private knowledge means that $v < 1$, and increasing social marginal productivity implies $1 < v + \gamma$. Note that $f(k,K) = F(k,K,\overline{x})$ for some fixed $\overline{x}$, and this functional is normalized with respect to $\overline{x}$. 

Romer uses the following simple functional form for $g(I)$, which satisfies $g(I) < \alpha$ for all $t$ and $Dg(0) = 1$:
\[g(I) = \frac{\alpha I}{\alpha + I}\]

In the first example, he used the utility function $U(c_t) = \log(U(c_t))$ and for the second example, he used the linear function $U(c_t) = c_t$. In the third example, he assumes linear utility and examined the permanent effects of shocks to the economy at equilibrium. Romer proposes that cross-country differences may be examined from the perspective of the last example. For instance, imagine two countries with different initial knowledge (or two countries with the same initial knowledge, but one received an external shock), each with $N$ firms and $N$ consumers described above. Even though growth rate of knowledge, $g$, will approach $\alpha$ in both countries, assuming there is no dissemination of knowledge between the countries (each country has its own $K$ and "public knowledge" is only "public" within that economy), then larger country will always be more productive due to increasing return on knowledge. For simplicity and generality, we will assume the functional form $U(c_t) = \frac{c_{t}^{1-\sigma} - 1}{1-\sigma}$, which covers the first and second examples, and then analyze its dynamic, which will cover Romer's third example. To guarantee that the consumer's life-time utility converges, we will assume that the discount rate $\beta$ is sufficiently large such that $\alpha(v+\gamma) < \beta$. 

Then, we solve the following problem: 
\begin{equation*}
\begin{aligned}
& \underset{\{c_t\}}{\text{max}}
& & \sum\limits_{n=0}^{\infty} \beta^{t} \frac{c_{t}^{1-\sigma} - 1}{1-\sigma} \\
& \text{subject to}
& & k_{t+1} = k_t(1+\frac{\alpha (N^\gamma k^{v+\gamma} - c_t)}{\alpha+ N^\gamma k^{v+\gamma} - c_t})
\end{aligned}
\end{equation*}


\section{Extended Model}

We introduce a new factor of production, $H_t$, which stands for human capital of firm or the whole economy? If the latter, do we assume all firms have access to it? If the former, do we assume this to be constant across all firms? How to justify this, and what is initial human capital?

The model remains the same except the production function $F$ and investment function $G$ are modified as follows. $F$ now depends on $k_t$, $K_t$, $H_t$ -- do we keep $x$ as placeholder or just eliminates it? - If we want to test with data then we need to eliminate it, but this seems far-fetched. Should we still normalize k as Romer does, again if it is normalized, we can not use data unless the data is a proxy. 

$F$ is exhibits decreasing returns with respect to $H_t$, 

A competitive equilibrium must still exist under our modificiations, as they do not violate any assumptions used in Romer's proofs. By introducing $H_t$, we merely specified one of the variables in $x$ in Romer's model. As $H_t$ demonstrates decreasing returns, just as the variables in $x$, no assumption about the production function is altered. The other modification, namely endogenizing the limit of $g$, also does not affect the existence of a competitive equilibrium. Romer required that $g$ always be bounded by some positive number, and our model binds $g$ such that it is always less than than the marginal return of human capital, which is a finite, positive number. -- Is this a mismatch? g is a rate and marginal return is a raw number, of course g is also a value. How do we do this to make intuitive sense? saying g is less than marginal rate seems really strict, so long g is finite everything is fine. 

\section{Data and Methodology}

\section{Results}


\section{References}
Romer, Paul M. (1986), "Increasing Returns and Long-Run Growth," \textit{The Journal of Political Economy}, Vol. 94, No. 5. (Oct., 1986), pp. 1002-1037.
\end{document}
